\documentclass[conference]{IEEEtran}
\IEEEoverridecommandlockouts
% The preceding line is only needed to identify funding in the first footnote. If that is unneeded, please comment it out.
\usepackage{cite}
\usepackage{amsmath,amssymb,amsfonts}
\usepackage{algorithmic}
\usepackage{graphicx}
\usepackage{textcomp}
\usepackage{xcolor}
\def\BibTeX{{\rm B\kern-.05em{\sc i\kern-.025em b}\kern-.08em
    T\kern-.1667em\lower.7ex\hbox{E}\kern-.125emX}}
\begin{document}

\title{A fuzzy inference model for cancer risk prediction\\
}

\author{\IEEEauthorblockN{María Camila Vásquez Correa}
\IEEEauthorblockA{\textit{Mathematics department} \\
\textit{Universidad EAFIT}\\
Medellín, Colombia \\
mvasqu49@eafit.edu.co}
}

\maketitle

\begin{abstract}
A significant number of people pass away due to limited medical resources for the battle with cancer. Fatal cases can be reduced by using the computational techniques in the medical and health system. If the cancer is diagnosed early, the chance of successful treatment increases. In
In this article is presented a model of fuzzy inference in cancer risk, with three variables of support: habits, heritage and age. The model is tested in various scenarios with different methods for logical operations and defuzzification. 
\end{abstract}

\begin{IEEEkeywords}
Fuzzy, Inference, FIS, cancer, risk.
\end{IEEEkeywords}

\section{Introduction}
Cancer is the leading life-threatening disease for people in today’s world. Although cancer formation is different and doctors often cannot explain why one person develops cancer and another does not, research shows that certain risk factors increase the chance that a person will develop cancer \cite{COGGON20051434}. Cancer prevention through consequent screening programs, early discovery and timely, improved and diversified means of treatment are usually the most successful ways to reduce mortality. 
Where uncertainty exists such as in the medical
field, fuzzy logic could play an important role in making decisions. Fuzzy logic is the science of reasoning, thinking, and inference that recognizes and uses the real world phenomenon that every- thing is a matter of degree. In the simplest terms, fuzzy logic theory is an extension of binary theory that does not use crisp definitions and distinctions.
Instead of assuming everything must be
defined crisply into black and white (binary view), fuzzy logic is a method that captures and uses the concept of fuzziness in a computationally effective manner \cite{Zadeh1996}. 

\section{Literature review}

\section{Methodology}

\section{Results}

\section{Conclusions}

\section*{References}

\nocite{*}
\bibliography{ref}
\bibliographystyle{IEEEtran}


\end{document}
